\newcommand{\AND}{\unskip
\cleaders\copy\ANDbox\hskip\wd\ANDbox
\ignorespaces
}
\newsavebox\ANDbox
\sbox\ANDbox{$|$}

\begin{header}
\fontsize{20 pt}{20 pt}\selectfont Luan Willig Silveira

\fontsize{12 pt}{12 pt}\selectfont \textit{Programador Full Stack}

\vspace{5 pt}

\normalsize
\kern 5.0 pt%
\mbox{\hrefWithoutArrow{mailto:luan.w.silveira@gmail.com}{{\footnotesize\faEnvelope[regular]}\hspace*{0.13cm}luan.w.silveira@gmail.com}}%
\kern 5.0 pt%
\AND%
\kern 5.0 pt%
\mbox{\hrefWithoutArrow{tel:+55 51 99502-6527}{{\footnotesize\faPhone*}\hspace*{0.13cm}+55 (51) 99502-6527}}%
\kern 5.0 pt%
\AND%
\kern 5.0 pt%
\mbox{\hrefWithoutArrow{https://linkedin.com/in/luanws}{{\footnotesize\faLinkedinIn}\hspace*{0.13cm}linkedin.com/in/luanws}}%
\kern 5.0 pt%
\AND%
\kern 5.0 pt%
\mbox{\hrefWithoutArrow{https://github.com/luanws}{{\footnotesize\faGithub}\hspace*{0.13cm}github.com/luanws}}%
\end{header}

\vspace{5 pt - 0.3 cm}


\section{Perfil}

\begin{onecolentry}
    Sou programador full stack desde 2019, formado em Engenharia Elétrica pela Universidade Federal de Santa Maria, com ampla experiência no desenvolvimento de software para Backend, Frontend WEB e Mobile. Adquiri experiência em todas as etapas do ciclo de desenvolvimento de software, desde o planejamento, análise, codificação, testes e deploy. Atuei como desenvolvedor de software em empresas, além de freelancers, iniciativas pessoais e atividades de pesquisa e extensão universitária.
\end{onecolentry}



\section{Experiência profissional}

\begin{twocolentry}{Jan 2024 – Atual}
\textbf{Programador}, WEG\end{twocolentry}
\vspace{0.10 cm}
\begin{onecolentry}
    Desenvolvimento backend com arquitetura de microsserviços utilizando Java, Spring Boot, MongoDB e PostgreSQL, e frontend web com JavaScript, TypeScript e Next.js, Tailwind.
    \begin{highlights}
        \item Melhorei tempo de resposta de uma aplicação em 10x utilizando Redis e cache;
        \item Criei uma API completa para internacionalização;
        \item Desenvolvi uma API para geração de relatórios em PDF com Java, Spring Boot, MongoDB e jsreport;
        \item Implementei um sistema de mensagens assíncronas utilizando Webhook, SSE, SQS e SNS.
    \end{highlights}
    
\end{onecolentry}


\vspace{0.2 cm}

\begin{twocolentry}{Jun 2023 – Atual}
\textbf{Programador e Pesquisador}, CPFL\end{twocolentry}
\vspace{0.10 cm}
\begin{onecolentry}
    Trabalhei como full stack desenvolvendo ferramentas de gerenciamento de datasets, processamento de imagem e modelos de IA com Python, TypeScript, Next.js, Java, Spring Boot e MongoDB, atuando em projeto de pesquisa aplicada no meu mestrado, com o objetivo de criar um sistema automatizado para detecção de falhas em linhas de transmissão com drones e IA.
\end{onecolentry}


\vspace{0.2 cm}

\begin{twocolentry}{Jan 2019 – Jan 2024}
\textbf{Trabalho como freelancer}\end{twocolentry}
\vspace{0.10 cm}
\begin{onecolentry}
    Trabalhei como desenvolvedor full stack independente, atuando em todas as etapas do desenvolvimento de um software, desde o planejamento até o deploy. Alguns dos projetos que desenvolvi são:
    \begin{highlights}
        \item Sistema de gerenciamento de chatbots para WhatsApp utilizando Inteligência Artificial (JavaScript, TypesScript, Node.js, Prisma ORM, PostgreSQL, Fastify, Next.js, React, Tailwind, n8n, Pipeline, Docker);
        \item Guia telefônico virtual onde usuários criam seus próprios contatos composto por backend, frontend web, mobile e plataforma desktop (Python, Flask, PostgreSQL, TypeScript, React, React Native, Expo);
        \item Aplicativo mobile que monitora a localização dos entregadores de jornais e notifica ao cliente quando o jornal está próximo de ser entregue (TypeScript, Express.js, PostgreSQL, React Native, Expo).
    \end{highlights}
\end{onecolentry}


\vspace{0.2 cm}

\begin{twocolentry}{Jan 2019 – Jan 2024}
\textbf{Projetos na Universidade}\end{twocolentry}
\vspace{0.10 cm}
\begin{onecolentry}
    Durante minha graduação, realizei diversas atividades complementares. Algumas delas são:
    \begin{highlights}
        \item Ministrei um curso de introdução à programação;
        \item Desenvolvi uma plataforma WEB capaz de recomendar recursos de aprendizagem para estudantes com base em filtragem colaborativa (Python, Flask, TypeScript, Angular, PostgreSQL);
        \item Trabalhei na criação de jogos educativos com foco em aprimoramento do raciocínio lógico, cálculo e memória do jogador (Python, C\#, JavaScript, Unity, Firebase, PostgreSQL).
    \end{highlights}
\end{onecolentry}



\section{Formação}

\begin{twocolentry}{2024 – Atual}
\textbf{Universidade Federal de Santa Maria}, Mestrado em Engenharia Elétrica\end{twocolentry}

\vspace{0.10 cm}
\begin{twocolentry}{2017 – 2023}
\textbf{Universidade Federal de Santa Maria}, Bacharelado em Engenharia Elétrica\end{twocolentry}



\section{Tecnologias}

\begin{onecolentry}
    \textbf{Linguagens:} Java, JavasScript, TypeScript, Python, C\#, Dart, HTML, CSS, SQL
\end{onecolentry}

\vspace{0.2 cm}

\begin{onecolentry}
    \textbf{Bancos de dados:} PostgreSQL, MongoDB, SQLite, Redis, Firebase Realtime Database, Firestore
\end{onecolentry}

\vspace{0.2 cm}

\begin{onecolentry}
    \textbf{Frameworks:} Spring Boot, Next.js, React, React Native, Flask, Express.js, Unity, Flutter
\end{onecolentry}

\vspace{0.2 cm}

\begin{onecolentry}
    \textbf{Outras tecnologias:} Git, Docker, Expo, Firebase, Supabase, AWS, GCP (Google Cloud Platform)
\end{onecolentry}
